\section{Partition\-Stats Class Reference}
\label{classPartitionStats}\index{PartitionStats@{PartitionStats}}
Implements operations between multiple partitions.  


{\tt \#include $<$Partition\-Stats.h$>$}

\subsection*{Public Member Functions}
\begin{CompactItemize}
\item 
{\bf Partition\-Stats} ()\label{classPartitionStats_a0}

\begin{CompactList}\small\item\em Default constructor simply resets all members to \char`\"{}zero\char`\"{}. \item\end{CompactList}\item 
{\bf Partition\-Stats} (vector$<$ {\bf Charr} $>$ fnames, part\-File\-Format iformat=part\-Fmt\-PART, double extensivity=EXTENSIVITY\_\-DEFAULT, int ofs=CLUSTEROFFSET\_\-DEFAULT, int clstat\_\-normalization\_\-ofs=0, char $\ast$mcltabfile=NULL)\label{classPartitionStats_a1}

\begin{CompactList}\small\item\em Instantiates a Partition\-Stats out of a list of filenames, a common file input format, a float parameter, a common cluster-offset value and a common normalization factor gauge (default=0). \item\end{CompactList}\item 
{\bf Partition\-Stats} (vector$<$ {\bf Charr} $>$ fnames, char $\ast$mcltabfile=NULL, double extensivity=EXTENSIVITY\_\-DEFAULT)\label{classPartitionStats_a2}

\begin{CompactList}\small\item\em Instantiates a Partition\-Stats out of a list of filenames, a common file tabfile, and a float parameter used for non-standard entropies. \item\end{CompactList}\item 
{\bf Partition\-Stats} (vector$<$ Partition $>$ \&vpart, double extensivity=EXTENSIVITY\_\-DEFAULT)\label{classPartitionStats_a3}

\begin{CompactList}\small\item\em Instantiates from a vector of partitions. \item\end{CompactList}\item 
void {\bf define\-List\-Of\-Partitions} (vector$<$ {\bf Charr} $>$ fnames, char $\ast$mcltabfile=NULL, part\-File\-Format iformat=part\-Fmt\-PART, int ofs=CLUSTEROFFSET\_\-DEFAULT)\label{classPartitionStats_a4}

\begin{CompactList}\small\item\em Declares a second list of input partition files, or the initial one, if it hasn't been declared before. \item\end{CompactList}\item 
bool {\bf clear2nd\-List\-Of\-Partitions} ()\label{classPartitionStats_a5}

\begin{CompactList}\small\item\em Erases second list of input partition files. \item\end{CompactList}\item 
bool {\bf has2nd\-List\-Of\-Partitions} ()\label{classPartitionStats_a6}

\begin{CompactList}\small\item\em Checks whether a second list of input partitions file has been defined. \item\end{CompactList}\item 
int {\bf are\-Partitions} ()\label{classPartitionStats_a7}

\begin{CompactList}\small\item\em Checks if each of the provided partitions is a sound partition, i.e., if all clusters are pair-wise disjoint. \item\end{CompactList}\item 
void {\bf get\-Cover} ()\label{classPartitionStats_a8}

\begin{CompactList}\small\item\em Builds the cover \_\-cover of the underlying set of elements out of the list of partitions. \item\end{CompactList}\item 
bool {\bf hasa\-Cover} ()\label{classPartitionStats_a9}

\begin{CompactList}\small\item\em Check if \_\-cover has been build. \item\end{CompactList}\item 
void {\bf print\-Cover} (bool SETUPCONSENSUSP)
\begin{CompactList}\small\item\em Prints out \_\-cover, eventually printing out also the consensus if SETUPCONSENSUSP=true within the same loop. Notice this doesn't instantiates a consensus partition, but just prints it. \item\end{CompactList}\item 
void {\bf print\-Consensus\-Part} ()\label{classPartitionStats_a11}

\begin{CompactList}\small\item\em Used by print\-Cover to print the consensus partition. \item\end{CompactList}\item 
Partition {\bf get\-Consensus\-Partition} ()
\begin{CompactList}\small\item\em Explicitly build and instantiate a new partition which is the consensus one of the given list of partitions. \item\end{CompactList}\item 
void {\bf set\-Consensus\_\-Ad} ()\label{classPartitionStats_a13}

\begin{CompactList}\small\item\em Build consensus adjacency matrix. \item\end{CompactList}\item 
graph {\bf Ad} ()\label{classPartitionStats_a14}

\begin{CompactList}\small\item\em Returns consensus adjacency matrix. \item\end{CompactList}\item 
void {\bf pgm\_\-Ad} ()\label{classPartitionStats_a15}

\begin{CompactList}\small\item\em Print grey-scale image (pgm) of consensus adjacency matrix. \item\end{CompactList}\item 
void {\bf get\_\-Ad} ()\label{classPartitionStats_a16}

\begin{CompactList}\small\item\em Print consensus adjacency matrix in raw format: list of rows for each edge : str\-A str\-B double. \item\end{CompactList}\item 
void {\bf get\_\-Fuzzy\-Consensus\-Partition} ()\label{classPartitionStats_a17}

\begin{CompactList}\small\item\em Print fuzzy partition associated to the consensus adjacency matrix. \item\end{CompactList}\item 
void {\bf get\-Ad\-Cos} ()\label{classPartitionStats_a18}

\begin{CompactList}\small\item\em Calculate cosine distance between associated Adjacency matrices. \item\end{CompactList}\item 
void {\bf get\-Purity} ()\label{classPartitionStats_a19}

\begin{CompactList}\small\item\em Calculate all pair-wise purity scores. \item\end{CompactList}\item 
void {\bf get\-Purity\-Ref} ()\label{classPartitionStats_a20}

\begin{CompactList}\small\item\em Calculate purity scores of all againts the first partition (reference). \item\end{CompactList}\item 
void {\bf get\-Purity\-Target} ()\label{classPartitionStats_a21}

\begin{CompactList}\small\item\em Calculate purity scores of this partition (target) againts all other partition. \item\end{CompactList}\item 
void {\bf get\-Splits\-Ref} (splitmethod similarity)\label{classPartitionStats_a22}

\begin{CompactList}\small\item\em Compare cluster by cluster against the first partition's clusters (reference) and print overlap (fraction elements present in referece cluster). \item\end{CompactList}\item 
long int {\bf card} (Partition p)\label{classPartitionStats_a23}

\begin{CompactList}\small\item\em get cardinality of partition \item\end{CompactList}\item 
double {\bf H} (Partition p)\label{classPartitionStats_a24}

\begin{CompactList}\small\item\em Get Shannon entropy of partition. \item\end{CompactList}\item 
double {\bf BK} (Partition p)\label{classPartitionStats_a25}

\begin{CompactList}\small\item\em Get Boltzman entropy of partition. \item\end{CompactList}\item 
double {\bf TS} (Partition p, double q)\label{classPartitionStats_a26}

\begin{CompactList}\small\item\em Get Tsallis entropy of partition. \item\end{CompactList}\item 
double {\bf TS} (Partition p)\label{classPartitionStats_a27}

\item 
double {\bf RS} (Partition p, double q)\label{classPartitionStats_a28}

\begin{CompactList}\small\item\em Get Renyi entropy of partition. \item\end{CompactList}\item 
double {\bf RS} (Partition p)\label{classPartitionStats_a29}

\item 
double {\bf JQnorm} (Partition p, double q)\label{classPartitionStats_a30}

\begin{CompactList}\small\item\em Get Jeffreys Qnorm (a la Tarantola) of partition. \item\end{CompactList}\item 
void {\bf i\-Potential} (pmetricv pm)\label{classPartitionStats_a31}

\begin{CompactList}\small\item\em Print the value of the (information theoretic) potential associated with each partition. \item\end{CompactList}\item 
long int {\bf ES} (Partition \&p1, Partition \&p2)\label{classPartitionStats_a32}

\begin{CompactList}\small\item\em Calculates edit score distance between part1 and part2 using pmetric. \item\end{CompactList}\item 
double {\bf VI} (Partition \&p1, Partition \&p2)\label{classPartitionStats_a33}

\begin{CompactList}\small\item\em Calculates VI (Shannon) distance between part1 and part2 using pmetric. Should give the same as using partition explicitly built-in vipp function. Already checked? \item\end{CompactList}\item 
double {\bf BK} (Partition \&p1, Partition \&p2)\label{classPartitionStats_a34}

\begin{CompactList}\small\item\em Calculates Boltzman distance between part1 and part2 using pmetric. \item\end{CompactList}\item 
double {\bf TS} (Partition \&p1, Partition \&p2)\label{classPartitionStats_a35}

\begin{CompactList}\small\item\em Calculates Tsallis distance between part1 and part2 using pmetric. \item\end{CompactList}\item 
double {\bf RS} (Partition \&p1, Partition \&p2)\label{classPartitionStats_a36}

\begin{CompactList}\small\item\em Calculates Renyi distance between part1 and part2 using pmetric. \item\end{CompactList}\item 
double {\bf TD} (Partition \&p1, Partition \&p2)\label{classPartitionStats_a37}

\begin{CompactList}\small\item\em Calculates Tarantola distance between part1 and part2 using Jeffrey's Qnorm. \item\end{CompactList}\item 
void {\bf distances} (Partition \&p, pmetricv pm=shannon)\label{classPartitionStats_a38}

\begin{CompactList}\small\item\em Calculates all distances againts the specified reference partition. Argument pmetricv specifies which metric to use (VI, Edit score,...). \item\end{CompactList}\item 
void {\bf distances\-Ref} (pmetricv pm=shannon)\label{classPartitionStats_a39}

\begin{CompactList}\small\item\em Calculates all distances againts the reference partition. This is the first partition read. Argument pmetricv specifies which metric to use (VI, Edit score,...). \item\end{CompactList}\item 
void {\bf distances} (pmetricv pm=shannon)\label{classPartitionStats_a40}

\begin{CompactList}\small\item\em Calculates all pair-wise distances. Argument pmetricv specifies which metric to use (VI, Edit score,...). \item\end{CompactList}\item 
{\bf Sampling} {\bf distances\-Distribution} (pmetricv pm=shannon, int refindex=-1)
\item 
void {\bf distances\_\-Subsprojection} (pmetricv pm=shannon)
\item 
void {\bf distances\-Ref\_\-Subsprojection} (pmetricv pm=shannon)\label{classPartitionStats_a43}

\begin{CompactList}\small\item\em Equivalent one for distances againts a common reference partition. \item\end{CompactList}\item 
void {\bf pmeasures} (pmeasure measure, pmetricv pm=shannon)\label{classPartitionStats_a44}

\begin{CompactList}\small\item\em Calculates different symmetric and non-symmetric pair-wise measures. \item\end{CompactList}\item 
void {\bf pmeasures\-Ref} (pmeasure measure, pmetricv pm=shannon)\label{classPartitionStats_a45}

\begin{CompactList}\small\item\em Calculates different symmetric and non-symmetric pair-wise measures against a common reference partition. \item\end{CompactList}\item 
void {\bf print\-Hasse\-Diagram} ()\label{classPartitionStats_a46}

\begin{CompactList}\small\item\em Prints the Hasse diagram corresponding to the given list of partitions. \item\end{CompactList}\item 
void {\bf print\-Hasse\-Nodes} ()\label{classPartitionStats_a47}

\begin{CompactList}\small\item\em Print. \item\end{CompactList}\end{CompactItemize}
\subsection*{Public Attributes}
\begin{CompactItemize}
\item 
{\bf Bell\-Number} {\bf Bell\-N}\label{classPartitionStats_o0}

\end{CompactItemize}


\subsection{Detailed Description}
Implements operations between multiple partitions. 

Implements operations between multiple partitions. Calculates different kind of pair-wise potentials, distances, inequality-residuals, the cover, ... 



\subsection{Member Function Documentation}
\index{PartitionStats@{Partition\-Stats}!distances_Subsprojection@{distances\_\-Subsprojection}}
\index{distances_Subsprojection@{distances\_\-Subsprojection}!PartitionStats@{Partition\-Stats}}
\subsubsection{\setlength{\rightskip}{0pt plus 5cm}void Partition\-Stats::distances\_\-Subsprojection (pmetricv {\em pm} = shannon)}\label{classPartitionStats_a42}


Aproximate calculation of pair-wise distance between paritions with different number of elements It stripps of all elements that aren't share and calculates the distance using simply the rest. \index{PartitionStats@{Partition\-Stats}!distancesDistribution@{distancesDistribution}}
\index{distancesDistribution@{distancesDistribution}!PartitionStats@{Partition\-Stats}}
\subsubsection{\setlength{\rightskip}{0pt plus 5cm}{\bf Sampling} Partition\-Stats::distances\-Distribution (pmetricv {\em pm} = shannon, int {\em refindex} = -1)}\label{classPartitionStats_a41}


Meadian, avg, std, std.err, var, min , max and sample sizeof the pair-wise distances among its partitions. If a non negative index is provided, distances are calculated with respect to partition refindex\index{PartitionStats@{Partition\-Stats}!getConsensusPartition@{getConsensusPartition}}
\index{getConsensusPartition@{getConsensusPartition}!PartitionStats@{Partition\-Stats}}
\subsubsection{\setlength{\rightskip}{0pt plus 5cm}Partition Partition\-Stats::get\-Consensus\-Partition ()}\label{classPartitionStats_a12}


Explicitly build and instantiate a new partition which is the consensus one of the given list of partitions. 

ccl is a simple map with a string as key. Thus each neighbors set appears only once.

If the map increased in size it means that the last set of neighbors is a new one

update then the counter and add this set to the consensus partition. \index{PartitionStats@{Partition\-Stats}!printCover@{printCover}}
\index{printCover@{printCover}!PartitionStats@{Partition\-Stats}}
\subsubsection{\setlength{\rightskip}{0pt plus 5cm}void Partition\-Stats::print\-Cover (bool {\em PRINTCONSENSUP} = false)}\label{classPartitionStats_a10}


Prints out \_\-cover, eventually printing out also the consensus if SETUPCONSENSUSP=true within the same loop. Notice this doesn't instantiates a consensus partition, but just prints it. 

ccl is a simple map with a string as key. Thus each neighbors set appears only once.

If the map increased in size it means that the last set of neighbors is a new one

update then the counter and add this set to the consensus partition. 

The documentation for this class was generated from the following files:\begin{CompactItemize}
\item 
Partition\-Stats.h\item 
Partition\-Stats.cc\end{CompactItemize}
