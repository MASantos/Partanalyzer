\section{Partition\-Stats Class Reference}
\label{classPartitionStats}\index{PartitionStats@{PartitionStats}}
\subsection*{Public Member Functions}
\begin{CompactItemize}
\item 
{\bf Partition\-Stats} (vector$<$ {\bf Charr} $>$ fnames, part\-File\-Format iformat, int ofs, int clstat\_\-normalization\_\-ofs)\label{classPartitionStats_a1}

\begin{CompactList}\small\item\em Instantiates a Partition\-Stats out of a list of filenames, a common file input format, a common cluster-offset value and a common normalization factor gauge (default=0). \item\end{CompactList}\item 
void {\bf get\-Cover} ()\label{classPartitionStats_a2}

\begin{CompactList}\small\item\em Builds the cover \_\-cover of the underlying set of elements out of the list of partitions. \item\end{CompactList}\item 
bool {\bf hasa\-Cover} ()\label{classPartitionStats_a3}

\begin{CompactList}\small\item\em Check if \_\-cover has been build. \item\end{CompactList}\item 
void {\bf print\-Cover} (bool SETUPCONSENSUSP)
\begin{CompactList}\small\item\em Prints out \_\-cover, eventually printing out also the consensus if SETUPCONSENSUSP=true within the same loop. Notice this doesn't instantiates a consensus partition, but just prints it. \item\end{CompactList}\item 
void {\bf print\-Consensus\-Part} ()\label{classPartitionStats_a5}

\begin{CompactList}\small\item\em Used by print\-Cover to print the consensus partition. \item\end{CompactList}\item 
{\bf Partition} {\bf get\-Consensus\-Partition} ()
\begin{CompactList}\small\item\em Explicitly build an instantiate a new partition which is the consensus one of the given list of partitions. \item\end{CompactList}\item 
int {\bf are\-Partitions} ()\label{classPartitionStats_a7}

\begin{CompactList}\small\item\em Checks if each of the provided partitions is a sound partition, i.e., if all clusters are pair-wise disjoint. \item\end{CompactList}\item 
void {\bf get\-Purity} ()\label{classPartitionStats_a8}

\begin{CompactList}\small\item\em Calculate all pair-wise purity scores. \item\end{CompactList}\item 
void {\bf get\-Purity\-Ref} ()\label{classPartitionStats_a9}

\begin{CompactList}\small\item\em Calculate purity scores of all againts the first partition (reference). \item\end{CompactList}\item 
double {\bf H} ({\bf Partition} p)\label{classPartitionStats_a10}

\begin{CompactList}\small\item\em Get entropy of partition. \item\end{CompactList}\item 
long int {\bf card} ({\bf Partition} p)\label{classPartitionStats_a11}

\begin{CompactList}\small\item\em get cardinality of partition \item\end{CompactList}\item 
double {\bf VI} ({\bf Partition} \&p1, {\bf Partition} \&p2)\label{classPartitionStats_a12}

\begin{CompactList}\small\item\em Calculates VI distance between part1 and part2 using pmetric. Should give the same as using partition explicitly built-in vipp function. Already checked? \item\end{CompactList}\item 
void {\bf distances} ({\bf Partition} \&p, pmetricv pm)\label{classPartitionStats_a13}

\begin{CompactList}\small\item\em Calculates all distances againts the specified reference partition. Argument pmetricv specifies which metric to use (VI, Edit score,...). \item\end{CompactList}\item 
void {\bf distances\-Ref} (pmetricv pm)\label{classPartitionStats_a14}

\begin{CompactList}\small\item\em Calculates all distances againts the reference partition. This is the first partition read. Argument pmetricv specifies which metric to use (VI, Edit score,...). \item\end{CompactList}\item 
void {\bf distances} (pmetricv pm)\label{classPartitionStats_a15}

\begin{CompactList}\small\item\em Calculates all pair-wise distances. Argument pmetricv specifies which metric to use (VI, Edit score,...). \item\end{CompactList}\item 
long int {\bf ES} ({\bf Partition} \&p1, {\bf Partition} \&p2)\label{classPartitionStats_a16}

\begin{CompactList}\small\item\em Calculates edit score distance between part1 and part2 using pmetric. \item\end{CompactList}\item 
void {\bf print\-Hasse\-Diagram} ()\label{classPartitionStats_a17}

\begin{CompactList}\small\item\em Prints the Hasse diagram corresponding to the given list of partitions. \item\end{CompactList}\item 
void {\bf print\-Hasse\-Nodes} ()\label{classPartitionStats_a18}

\begin{CompactList}\small\item\em Print. \item\end{CompactList}\end{CompactItemize}
\subsection*{Public Attributes}
\begin{CompactItemize}
\item 
{\bf Bell\-Number} {\bf Bell\-N}\label{classPartitionStats_o0}

\end{CompactItemize}


\subsection{Detailed Description}
Implements operations between multiple partitions. The most important one is determining the consensus partition among a given set of partitions of a same set X 



\subsection{Member Function Documentation}
\index{PartitionStats@{Partition\-Stats}!getConsensusPartition@{getConsensusPartition}}
\index{getConsensusPartition@{getConsensusPartition}!PartitionStats@{Partition\-Stats}}
\subsubsection{\setlength{\rightskip}{0pt plus 5cm}{\bf Partition} Partition\-Stats::get\-Consensus\-Partition ()}\label{classPartitionStats_a6}


Explicitly build an instantiate a new partition which is the consensus one of the given list of partitions. 

ccl is a simple map with a string as key. Thus each neighbors set appears only once.

If the map increased in size it means that the last set of neighbors is a new one

update then the counter and add this set to the consensus partition. \index{PartitionStats@{Partition\-Stats}!printCover@{printCover}}
\index{printCover@{printCover}!PartitionStats@{Partition\-Stats}}
\subsubsection{\setlength{\rightskip}{0pt plus 5cm}void Partition\-Stats::print\-Cover (bool {\em PRINTCONSENSUP} = false)}\label{classPartitionStats_a4}


Prints out \_\-cover, eventually printing out also the consensus if SETUPCONSENSUSP=true within the same loop. Notice this doesn't instantiates a consensus partition, but just prints it. 

ccl is a simple map with a string as key. Thus each neighbors set appears only once.

If the map increased in size it means that the last set of neighbors is a new one

update then the counter and add this set to the consensus partition. 

The documentation for this class was generated from the following file:\begin{CompactItemize}
\item 
partanalyze.cc\end{CompactItemize}
