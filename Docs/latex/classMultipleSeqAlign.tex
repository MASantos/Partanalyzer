\section{Multiple\-Seq\-Align Class Reference}
\label{classMultipleSeqAlign}\index{MultipleSeqAlign@{MultipleSeqAlign}}
{\tt \#include $<$Multiple\-Seq\-Align.h$>$}

\subsection*{Public Member Functions}
\begin{CompactItemize}
\item 
{\bf Multiple\-Seq\-Align} ()\label{classMultipleSeqAlign_a0}

\begin{CompactList}\small\item\em Default constructor. \item\end{CompactList}\item 
{\bf Multiple\-Seq\-Align} (char $\ast$msaf)
\begin{CompactList}\small\item\em Expects a MSA file in fasta format. \item\end{CompactList}\item 
{\bf Multiple\-Seq\-Align} (MSA \&msa)\label{classMultipleSeqAlign_a2}

\begin{CompactList}\small\item\em Expects a MSA object. \item\end{CompactList}\item 
{\bf Multiple\-Seq\-Align} {\bf xtract\-Positions} (vector$<$ int $>$ $\ast$positions=NULL)\label{classMultipleSeqAlign_a3}

\begin{CompactList}\small\item\em Subsample MSA at given positions. \item\end{CompactList}\item 
{\bf Multiple\-Seq\-Align} {\bf xtract\-Sequences} (svect $\ast$seqnames, bool equal=true)\label{classMultipleSeqAlign_a4}

\begin{CompactList}\small\item\em Extract sequences specified by name. IF bool equal=false, drop them instead. \item\end{CompactList}\item 
{\bf Multiple\-Seq\-Align} {\bf xtract\-Sequences\-By\-Id} ({\bf Multiple\-Seq\-Align} $\ast$msab, double min\-Id=30, double max\-Id=100, bool equal=true, vector$<$ int $>$ $\ast$positions=NULL)\label{classMultipleSeqAlign_a5}

\begin{CompactList}\small\item\em Extract non-redundant set of sequences with Id between min\-Id and max\-Id against Multiple sequence alignment msab. IF bool equal=false, drop them instead. \item\end{CompactList}\item 
{\bf Multiple\-Seq\-Align} {\bf xtract\-Sequences\-Highest\-Id} ({\bf Multiple\-Seq\-Align} $\ast$msab, int cullsize=0, vector$<$ int $>$ $\ast$positions=NULL)\label{classMultipleSeqAlign_a6}

\begin{CompactList}\small\item\em Extract max. cullsize nonredundant sequences most similar to those of MSA msab. If cullsize=0, cull as many as possible. \item\end{CompactList}\item 
{\bf Multiple\-Seq\-Align} {\bf drop\-Clones} ()\label{classMultipleSeqAlign_a7}

\begin{CompactList}\small\item\em Extract the NO-CLONE (non-redundant) subset of sequences. Requires fasta file, but sequences may not be aligned. \item\end{CompactList}\item 
{\bf Multiple\-Seq\-Align} {\bf drop\-Clones} ({\bf Multiple\-Seq\-Align} $\ast$msab)\label{classMultipleSeqAlign_a8}

\item 
void {\bf print\-With\-Cluster\-Labels} ({\bf Partition} $\ast$part, MSAformat fmt=msa\-Fmt\-NULL)\label{classMultipleSeqAlign_a9}

\begin{CompactList}\small\item\em Map partition clusters onto MSA. \item\end{CompactList}\item 
{\bf Partition} {\bf get\-Partition} (char $\ast$msaf, MSAformat fmt=msa\-Fmt\-NULL)\label{classMultipleSeqAlign_a10}

\begin{CompactList}\small\item\em Given a MSA with Clusters' labels, get the partition. Somehow, the reverse of print\-With\-Cluster\-Labels. \item\end{CompactList}\item 
MSA::iterator {\bf begin\-Seq} ()\label{classMultipleSeqAlign_a11}

\begin{CompactList}\small\item\em Get begin iterator to \_\-Seqlist. \item\end{CompactList}\item 
MSA::iterator {\bf end\-Seq} ()\label{classMultipleSeqAlign_a12}

\begin{CompactList}\small\item\em Get end iterator to \_\-Seqlist. \item\end{CompactList}\item 
int {\bf get\-Number\-Of\-Seq} ()\label{classMultipleSeqAlign_a13}

\begin{CompactList}\small\item\em Get the number of sequences in MSA. \item\end{CompactList}\item 
void {\bf add\-Seq} ({\bf Sequence} $\ast$)\label{classMultipleSeqAlign_a14}

\begin{CompactList}\small\item\em Allows adding a single {\bf Sequence} object to the multiple sequence alignment. \item\end{CompactList}\item 
void {\bf add\-Seq} ({\bf Sequence})\label{classMultipleSeqAlign_a15}

\item 
bool {\bf empty} ()\label{classMultipleSeqAlign_a16}

\begin{CompactList}\small\item\em Is MSA empty? \item\end{CompactList}\item 
void {\bf set\-Name} (string name)\label{classMultipleSeqAlign_a17}

\begin{CompactList}\small\item\em Sets its name. \item\end{CompactList}\item 
string {\bf get\-Name} ()\label{classMultipleSeqAlign_a18}

\begin{CompactList}\small\item\em Gets its name. \item\end{CompactList}\item 
void {\bf set\-File\-Name} (char $\ast$fname)
\item 
char $\ast$ {\bf get\-File\-Name} ()\label{classMultipleSeqAlign_a20}

\begin{CompactList}\small\item\em Gets its file name. \item\end{CompactList}\item 
double {\bf Seq\-Id} ({\bf Sequence} Seqa, {\bf Sequence} Seqb, vector$<$ int $>$ $\ast$positions=NULL)\label{classMultipleSeqAlign_a21}

\begin{CompactList}\small\item\em Obtains the sequence identity between the two provided Sequences using the specified positions. \item\end{CompactList}\item 
double {\bf Seq\-Id} (int Seqn, int Seqm, vector$<$ int $>$ $\ast$positions=NULL)\label{classMultipleSeqAlign_a22}

\begin{CompactList}\small\item\em Obtains the sequence identity between the two provided sequences refered by their indexes within the MSA. \item\end{CompactList}\item 
double {\bf average\-Id} (vector$<$ int $>$ $\ast$positions=NULL)\label{classMultipleSeqAlign_a23}

\begin{CompactList}\small\item\em Calculates the overall average sequence identity among all pair of sequences of the MSA. \item\end{CompactList}\item 
void {\bf print\-Pairwise\-Ids} (vector$<$ int $>$ $\ast$positions=NULL)\label{classMultipleSeqAlign_a24}

\begin{CompactList}\small\item\em Prints all pair-wise identities. \item\end{CompactList}\item 
void {\bf print\-Average\-Pairwise\-Ids} (double thr=50.0, vector$<$ int $>$ $\ast$positions=NULL)
\begin{CompactList}\small\item\em Prints the average pair-wise identity and the fraction of all pairs with identity above thr. Default thr=50. \item\end{CompactList}\item 
void {\bf print\-Pairwise\-Ids} ({\bf Multiple\-Seq\-Align} \&msa, vector$<$ int $>$ $\ast$positions=NULL)\label{classMultipleSeqAlign_a26}

\begin{CompactList}\small\item\em Given alternative MSA, prints all pair-wise identities. \item\end{CompactList}\item 
void {\bf print\-Average\-Pairwise\-Ids} ({\bf Multiple\-Seq\-Align} \&msa, double thr, vector$<$ int $>$ $\ast$positions=NULL)
\begin{CompactList}\small\item\em Given alternative MSA, prints the average pair-wise identity and the fraction of all pairs with identity above thr. Default thr=50. \item\end{CompactList}\item 
void {\bf print} (bool sorted=false)\label{classMultipleSeqAlign_a28}

\begin{CompactList}\small\item\em Print the whole multiple sequence alignment. \item\end{CompactList}\item 
{\bf Multiple\-Seq\-Align} {\bf gen\-Redundant\-MSA} (int n)\label{classMultipleSeqAlign_a29}

\begin{CompactList}\small\item\em Generates randomly n additional sequences each an exact copy of one of the original sequences. \item\end{CompactList}\item 
void {\bf Nsamples\-Red\-MSA} (int nsamples, int n)
\item 
void {\bf Nsamples\-Red\-MSA} (unsigned int seed, int nsamples, int n)\label{classMultipleSeqAlign_a31}

\begin{CompactList}\small\item\em The same as the previous one, but allows specifying a particular seed. \item\end{CompactList}\end{CompactItemize}
\subsection*{Friends}
\begin{CompactItemize}
\item 
ostream \& {\bf operator$<$$<$} (ostream \&os, {\bf Multiple\-Seq\-Align} \&msa)\label{classMultipleSeqAlign_n0}

\end{CompactItemize}


\subsection{Detailed Description}
Allows to perform operations on multiple sequence aligments. Its main member being of course the MSA object \_\-Seqlist which contains the list of all sequences and their names. 



\subsection{Constructor \& Destructor Documentation}
\index{MultipleSeqAlign@{Multiple\-Seq\-Align}!MultipleSeqAlign@{MultipleSeqAlign}}
\index{MultipleSeqAlign@{MultipleSeqAlign}!MultipleSeqAlign@{Multiple\-Seq\-Align}}
\subsubsection{\setlength{\rightskip}{0pt plus 5cm}Multiple\-Seq\-Align::Multiple\-Seq\-Align (char $\ast$ {\em msaf})}\label{classMultipleSeqAlign_a1}


Expects a MSA file in fasta format. 

and check whether it may belong or not to the same MSA

If the length fits, add the previous sequence Seq to the MSA \_\-Seqlist

The last sequence isn't followed by a line starting with \char`\"{}$>$\char`\"{} 

\subsection{Member Function Documentation}
\index{MultipleSeqAlign@{Multiple\-Seq\-Align}!NsamplesRedMSA@{NsamplesRedMSA}}
\index{NsamplesRedMSA@{NsamplesRedMSA}!MultipleSeqAlign@{Multiple\-Seq\-Align}}
\subsubsection{\setlength{\rightskip}{0pt plus 5cm}void Multiple\-Seq\-Align::Nsamples\-Red\-MSA (int {\em nsamples}, int {\em n})}\label{classMultipleSeqAlign_a30}


Generaes nsamples of MSAs each with redundancy of size n, i.e., each having n redundant sequences. The system clock is used for generating the seed of the random number generator \index{MultipleSeqAlign@{Multiple\-Seq\-Align}!printAveragePairwiseIds@{printAveragePairwiseIds}}
\index{printAveragePairwiseIds@{printAveragePairwiseIds}!MultipleSeqAlign@{Multiple\-Seq\-Align}}
\subsubsection{\setlength{\rightskip}{0pt plus 5cm}void Multiple\-Seq\-Align::print\-Average\-Pairwise\-Ids ({\bf Multiple\-Seq\-Align} \& {\em msa}, double {\em thr}, vector$<$ int $>$ $\ast$ {\em positions} = NULL)}\label{classMultipleSeqAlign_a27}


Given alternative MSA, prints the average pair-wise identity and the fraction of all pairs with identity above thr. Default thr=50. 

Print Average\-ID Std\-Dev. Variance Min Max \#edges$>$50\%Id edges$>$50 total\#edges \index{MultipleSeqAlign@{Multiple\-Seq\-Align}!printAveragePairwiseIds@{printAveragePairwiseIds}}
\index{printAveragePairwiseIds@{printAveragePairwiseIds}!MultipleSeqAlign@{Multiple\-Seq\-Align}}
\subsubsection{\setlength{\rightskip}{0pt plus 5cm}void Multiple\-Seq\-Align::print\-Average\-Pairwise\-Ids (double {\em thr} = 50.0, vector$<$ int $>$ $\ast$ {\em positions} = NULL)}\label{classMultipleSeqAlign_a25}


Prints the average pair-wise identity and the fraction of all pairs with identity above thr. Default thr=50. 

Print Average\-ID Std\-Dev. Variance Min Max \#edges$>$50\%Id edges$>$50 total\#edges \index{MultipleSeqAlign@{Multiple\-Seq\-Align}!setFileName@{setFileName}}
\index{setFileName@{setFileName}!MultipleSeqAlign@{Multiple\-Seq\-Align}}
\subsubsection{\setlength{\rightskip}{0pt plus 5cm}void Multiple\-Seq\-Align::set\-File\-Name (char $\ast$ {\em fname})\hspace{0.3cm}{\tt  [inline]}}\label{classMultipleSeqAlign_a19}


Sets its file name. This, being just a char$\ast$ can be used as an additional lable. That can be helpful with multiple sequence alignments generated on the fly, during the calculations. 

The documentation for this class was generated from the following files:\begin{CompactItemize}
\item 
Multiple\-Seq\-Align.h\item 
Multiple\-Seq\-Align.cc\end{CompactItemize}
