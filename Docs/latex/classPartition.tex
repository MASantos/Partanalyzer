\section{Partition Class Reference}
\label{classPartition}\index{Partition@{Partition}}
{\tt \#include $<$Partition.h$>$}

\subsection*{Public Member Functions}
\begin{CompactItemize}
\item 
{\bf Partition} ()
\begin{CompactList}\small\item\em Default Partition constructor. \item\end{CompactList}\item 
{\bf Partition} (char $\ast$file, part\-File\-Format iformat=part\-Fmt\-PART, int ofs=2)\label{classPartition_a1}

\begin{CompactList}\small\item\em Instantiate partition from file using format iformat and specified offset (applies only to partanalyzer's own format part\-Fmt\-PART). \item\end{CompactList}\item 
{\bf Partition} (smat $\ast$clustersl, int ofs, bool dosort=true, char $\ast$partf=NULL, char $\ast$tabf=NULL)\label{classPartition_a2}

\begin{CompactList}\small\item\em Create a partition out of a set of clusters. \item\end{CompactList}\item 
{\bf Partition} (set$<$ sset $\ast$ $>$ sclassp, int ofs=2, char $\ast$partf=NULL, char $\ast$tabf=NULL)
\item 
void {\bf summary} ()\label{classPartition_a4}

\begin{CompactList}\small\item\em Print summary of most import facts. \item\end{CompactList}\item 
{\bf Partition} {\bf xtract\-Elements} (svect $\ast$elements)\label{classPartition_a5}

\begin{CompactList}\small\item\em Extract selected elements. \item\end{CompactList}\item 
void {\bf xtr\-Cons\-Part} (multimap$<$ int, string, {\bf greater\-Than} $>$ cons\-Part, int ofs=1)\label{classPartition_a6}

\begin{CompactList}\small\item\em Create a consensus partition. \item\end{CompactList}\item 
sset {\bf get\-Items} ()\label{classPartition_a7}

\begin{CompactList}\small\item\em Get (and reset) the underlying set of elements sorted by alphanumerically. \item\end{CompactList}\item 
sset {\bf get\-Items} (cluster $\ast$cl)\label{classPartition_a8}

\begin{CompactList}\small\item\em Get an ordered set with all elements contained in (svect) cl. \{Doesn't affect Partition; should be external maybe\}. \item\end{CompactList}\item 
bool {\bf isa\-Partition\-Of} (svect \&sset\-Of\-Elements)
\begin{CompactList}\small\item\em Check it is a sound Partition of the specified set of elements. \item\end{CompactList}\item 
bool {\bf isa\-Partition} ()\label{classPartition_a10}

\begin{CompactList}\small\item\em Check it is a sound Partition of the default set of elements. \item\end{CompactList}\item 
void {\bf print\-Partition} (bool Sequential\-Cluster\-Names=false, string Cluster\-Prefix=\char`\"{}C\char`\"{})\label{classPartition_a11}

\begin{CompactList}\small\item\em Print partition using the default partanalyzer format. \item\end{CompactList}\item 
void {\bf print\-Partition} (part\-File\-Format iformat, bool Sequential\-Cluster\-Names=false, string Cluster\-Prefix=\char`\"{}C\char`\"{})\label{classPartition_a12}

\begin{CompactList}\small\item\em Print partition in the specified format. \item\end{CompactList}\item 
long int {\bf n\_\-clusters} ()\label{classPartition_a13}

\begin{CompactList}\small\item\em Get number of clusters (including singletons). \item\end{CompactList}\item 
long int {\bf n\_\-singletons} ()\label{classPartition_a14}

\begin{CompactList}\small\item\em Get number of singletons. \item\end{CompactList}\item 
long int {\bf n\_\-pairs} ()\label{classPartition_a15}

\begin{CompactList}\small\item\em Get number of pair clusters. \item\end{CompactList}\item 
long int {\bf n\_\-non\-Sing\-Clusters} ()\label{classPartition_a16}

\begin{CompactList}\small\item\em Get number of non-singletons clusters. \item\end{CompactList}\item 
long int {\bf n\_\-items} ()\label{classPartition_a17}

\begin{CompactList}\small\item\em Get number of elements. \item\end{CompactList}\item 
int {\bf cluster\_\-offset} ()\label{classPartition_a18}

\begin{CompactList}\small\item\em Get cluster offset (used only with partanalyzer's own format part\-Fmt\-PART). \item\end{CompactList}\item 
int {\bf largest\_\-Cluster} ()\label{classPartition_a19}

\begin{CompactList}\small\item\em Get index of largest cluster in (smat) Partition::smat. \item\end{CompactList}\item 
smat::iterator {\bf it\_\-largest\_\-Cluster} ()\label{classPartition_a20}

\begin{CompactList}\small\item\em Get iterator pointing to largest cluster in (smat) Partition::smat. \item\end{CompactList}\item 
void {\bf set\-Part\-Input\-Format} (part\-File\-Format iformat)\label{classPartition_a21}

\begin{CompactList}\small\item\em Set partition input file format. \item\end{CompactList}\item 
part\-File\-Format {\bf get\-Part\-Input\-Format} ()\label{classPartition_a22}

\begin{CompactList}\small\item\em Retrieve partition input file format. \item\end{CompactList}\item 
void {\bf mcl\-Tab\-File} (char $\ast$mcltabf)\label{classPartition_a23}

\begin{CompactList}\small\item\em Load MCL tab file. \item\end{CompactList}\item 
void {\bf tab\-File} (char $\ast$tabf)\label{classPartition_a24}

\begin{CompactList}\small\item\em Load a general tab file for changing name of elements. \item\end{CompactList}\item 
void {\bf swap\-Labels} (char $\ast$mcltabf=NULL)\label{classPartition_a25}

\begin{CompactList}\small\item\em Swap labels according to the provided tab file. \item\end{CompactList}\item 
double {\bf H} ()\label{classPartition_a26}

\begin{CompactList}\small\item\em Entropy of a partition. \item\end{CompactList}\item 
long int {\bf card} ()\label{classPartition_a27}

\begin{CompactList}\small\item\em Cardinality of a partition. \item\end{CompactList}\item 
double {\bf BK} ()\label{classPartition_a28}

\begin{CompactList}\small\item\em Boltzman Entropy: log (\#available states). \item\end{CompactList}\item 
double {\bf TS} (double q=EXTENSIVITY\_\-DEFAULT\_\-TSALLIS)\label{classPartition_a29}

\begin{CompactList}\small\item\em Tsallis Entropy: is equal to H iif q=1 ; q=non-extensivity coefficient. \item\end{CompactList}\item 
double {\bf RS} (double q=EXTENSIVITY\_\-DEFAULT\_\-RENYI)\label{classPartition_a30}

\begin{CompactList}\small\item\em Renyi Entropy: is equal to H iif q=1 ; q=non-extensivity coefficient. \item\end{CompactList}\item 
double {\bf JQnorm} (double q)\label{classPartition_a31}

\begin{CompactList}\small\item\em Calculates the Jeffrey Qnorm (a la Tarantola) of a partition= exp(RS(q)) or expr(TS(q)). \item\end{CompactList}\item 
void {\bf vipp} ({\bf Partition} $\ast$part2)
\begin{CompactList}\small\item\em Calculates and prints vi distance againts given partition part2. \item\end{CompactList}\item 
void {\bf edsc} ({\bf Partition} $\ast$part2)\label{classPartition_a33}

\begin{CompactList}\small\item\em Calculates and prints edit score distance againts given partition part2. \item\end{CompactList}\item 
void {\bf td} ({\bf Partition} $\ast$part2, double q)\label{classPartition_a34}

\begin{CompactList}\small\item\em Calculates the Tarantola distance against given partition part2. \item\end{CompactList}\item 
double {\bf purity\-Strict} ({\bf Partition} $\ast$part2)\label{classPartition_a35}

\begin{CompactList}\small\item\em Calculates the purity strict values againts the given partition p2. \item\end{CompactList}\item 
double {\bf purity\-Lax} ({\bf Partition} $\ast$part2)\label{classPartition_a36}

\begin{CompactList}\small\item\em Calculates the purity lax values againts the given partition p2. \item\end{CompactList}\item 
vector$<$ double $>$ {\bf purity\-Score} ({\bf Partition} $\ast$part2)
\begin{CompactList}\small\item\em Calculates both purity score values againts the given partition p2. \item\end{CompactList}\item 
void {\bf missing} (svect $\ast$items\_\-found)\label{classPartition_a38}

\item 
void {\bf missing} ()\label{classPartition_a39}

\item 
int {\bf get\-Cluster\-Size} (const string \&item)\label{classPartition_a40}

\begin{CompactList}\small\item\em Given an element, get the size of the cluster that contains it. \item\end{CompactList}\item 
string {\bf get\-Cluster\-Name} (string \&item)\label{classPartition_a41}

\begin{CompactList}\small\item\em Given an element, get the name of the cluster that contains it. \item\end{CompactList}\item 
string {\bf get\-Cluster\-Name} (int \&clidx)\label{classPartition_a42}

\begin{CompactList}\small\item\em Given an index from clusters, get the name of that cluster. \item\end{CompactList}\item 
string {\bf get\-Cluster\-Name} (svect \&cluster)\label{classPartition_a43}

\begin{CompactList}\small\item\em Given the cluster, get its name. \item\end{CompactList}\item 
int {\bf get\-Cluster\-Idx} (string \&item)\label{classPartition_a44}

\begin{CompactList}\small\item\em Given an element, get the index of the cluster that contains it. \item\end{CompactList}\item 
bool {\bf are\-Equiv} (string a, string b)\label{classPartition_a45}

\begin{CompactList}\small\item\em Two elements are equivalent if they belong to the same cluster. \item\end{CompactList}\item 
string {\bf are\-Within\-Same\-Cluster} (string ita, string itb)\label{classPartition_a46}

\begin{CompactList}\small\item\em Return name of cluster if true; x if false ; NAN1 (NAN2) if first (second) not found. \item\end{CompactList}\item 
char $\ast$ {\bf File\-Name} ()\label{classPartition_a47}

\begin{CompactList}\small\item\em Get the filename. \item\end{CompactList}\item 
svect {\bf get\-Cluster\-Of} (string item)\label{classPartition_a48}

\begin{CompactList}\small\item\em cluster is a typedef specific for Partition \item\end{CompactList}\item 
graph {\bf set\-Adjacency\-Matrix} ()\label{classPartition_a49}

\begin{CompactList}\small\item\em Build associated Adjacency matrix. \item\end{CompactList}\item 
graph {\bf set\-Adjacency\-Matrix\_\-os} ()\label{classPartition_a50}

\begin{CompactList}\small\item\em Build associated Adjacency matrix sorting elements by clusters. \item\end{CompactList}\item 
graph {\bf Ad} ()\label{classPartition_a51}

\begin{CompactList}\small\item\em Get Adjacency\-Matrix. \item\end{CompactList}\item 
graph {\bf Ad\_\-os} ()\label{classPartition_a52}

\begin{CompactList}\small\item\em Get Adjacency\-Matrix with indexes ordered as given by the partition. \item\end{CompactList}\item 
void {\bf Subs\-Project} (sset \&itemset)\label{classPartition_a53}

\begin{CompactList}\small\item\em Project a Partition of N elements onto a subspace of M$<$=N elements (active view). \item\end{CompactList}\item 
{\bf Partition} {\bf intersection} ({\bf Partition} $\ast$part2)
\begin{CompactList}\small\item\em Calculate the partition intersection of the present one and part2 (passed by pointer). \item\end{CompactList}\item 
{\bf Partition} {\bf intersection} ({\bf Partition} \&part2)\label{classPartition_a55}

\begin{CompactList}\small\item\em Calculate the partition intersection of the present one and part2 (passed by reference). \item\end{CompactList}\item 
bool {\bf less\-Than} ({\bf Partition} $\ast$part2)\label{classPartition_a56}

\begin{CompactList}\small\item\em Define Lattice preorder relation (as an interface). \item\end{CompactList}\item 
bool {\bf operator$<$=} ({\bf Partition} \&part2)\label{classPartition_a57}

\begin{CompactList}\small\item\em Define Lattice preorder relation (as a binary operator). \item\end{CompactList}\item 
bool {\bf operator$<$} ({\bf Partition} \&part2)\label{classPartition_a58}

\begin{CompactList}\small\item\em Define lattice strict preorder relation (as a binary operator). \item\end{CompactList}\item 
bool {\bf operator==} ({\bf Partition} \&part2)\label{classPartition_a59}

\begin{CompactList}\small\item\em Define equality operater. \item\end{CompactList}\item 
{\bf Partition} {\bf operator $\ast$} ({\bf Partition} \&part2)\label{classPartition_a60}

\begin{CompactList}\small\item\em Define intersetion of partitions ( as a binary operator ; arguments as refereces). \item\end{CompactList}\item 
{\bf Partition} {\bf operator $\ast$} ({\bf Partition} $\ast$part2)\label{classPartition_a61}

\begin{CompactList}\small\item\em Define intersetion of partitions ( as a binary operator ; arguments as pointers). \item\end{CompactList}\item 
{\bf Partition} {\bf operator+} ({\bf Partition} \&part2)
\begin{CompactList}\small\item\em (TO BE IMPLEMENTED) Define union of partitions (as a binary operator) \item\end{CompactList}\end{CompactItemize}
\subsection*{Public Attributes}
\begin{CompactItemize}
\item 
smat {\bf clusters}\label{classPartition_o0}

\begin{CompactList}\small\item\em Vector of clusters, each being a svect containing number of elements, name of cluster, and all its elements. \item\end{CompactList}\item 
sset {\bf sitems}\label{classPartition_o1}

\begin{CompactList}\small\item\em Set of elements (treated as strings; represents the underlying set of elements). \item\end{CompactList}\item 
sset {\bf ssingletons}\label{classPartition_o2}

\begin{CompactList}\small\item\em Sorted set of singleton elements. \item\end{CompactList}\end{CompactItemize}


\subsection{Detailed Description}
Main class Partition implements a partition and its algebra 



\subsection{Constructor \& Destructor Documentation}
\index{Partition@{Partition}!Partition@{Partition}}
\index{Partition@{Partition}!Partition@{Partition}}
\subsubsection{\setlength{\rightskip}{0pt plus 5cm}Partition::Partition ()}\label{classPartition_a0}


Default Partition constructor. 

Let's make sure we do not get any duplicated singleton. This is just a cheap patch for trying to speed up {\bf isa\-Partition()}. \index{Partition@{Partition}!Partition@{Partition}}
\index{Partition@{Partition}!Partition@{Partition}}
\subsubsection{\setlength{\rightskip}{0pt plus 5cm}Partition::Partition (set$<$ sset $\ast$ $>$ {\em sclassp}, int {\em ofs} = 2, char $\ast$ {\em partf} = NULL, char $\ast$ {\em tabf} = NULL)}\label{classPartition_a3}


Create a partition out of a set of pointers to sets (let's call that classes: so a set of pointers to classes...) Partition \_\-will\_\- be sorted by size of clusters and also within each cluster automatically.

\subsection{Member Function Documentation}
\index{Partition@{Partition}!intersection@{intersection}}
\index{intersection@{intersection}!Partition@{Partition}}
\subsubsection{\setlength{\rightskip}{0pt plus 5cm}{\bf Partition} Partition::intersection ({\bf Partition} $\ast$ {\em part2})}\label{classPartition_a54}


Calculate the partition intersection of the present one and part2 (passed by pointer). 

Will contain intersection between cla and clb

Set its size and name. These will be its first two entries

First, the name of the cluster

the name of the cluster

Second, the size of the cluster

Second, the size of the cluster

Third, the fake reverse-sorting label

Third, the fake reverse-sorting label

Reverse sort : Larger clusters firstl singletons last

REmove the first fake reverse-sorting label of each cluster

pointer refering to largest cluster \index{Partition@{Partition}!isaPartitionOf@{isaPartitionOf}}
\index{isaPartitionOf@{isaPartitionOf}!Partition@{Partition}}
\subsubsection{\setlength{\rightskip}{0pt plus 5cm}bool Partition::isa\-Partition\-Of (svect \& {\em svec\-Ofelements})}\label{classPartition_a9}


Check it is a sound Partition of the specified set of elements. 

One single clusters does make a sound partitions

For the moment, lets assume clusters are not ordered in descending order by size

SKIP check for singletons, but non-singletons ARE checked againts clb singletons!!

With the introduction of {\bf Partition::ssingletons}, no instantiated Partition can have the same singleton element more than once. Hence, there is no need in checking singletons againts singletons. Looping over them means that, on an Opteron 846 2GH, it takes 15sec to check a partition with 502 elements and on average 185 clusters and 95 singletons. Does skipping singletons helps to significantly improve this figure? Also, take svect cla= clusters[i]; out of the second loop (that was weird!). Ok, preliminary test shows an increase in speed of $\sim$28\%, i.e, 9.5sec/partition now versus 13.2sec before! (Compared both ways using a set of 19 partitions of those same sizes. \index{Partition@{Partition}!operator+@{operator+}}
\index{operator+@{operator+}!Partition@{Partition}}
\subsubsection{\setlength{\rightskip}{0pt plus 5cm}{\bf Partition} Partition::operator+ ({\bf Partition} \& {\em part2})}\label{classPartition_a62}


(TO BE IMPLEMENTED) Define union of partitions (as a binary operator) 

Start of largest cluster in this

and end.

extract only the elements, i.e., w/o the labels.

Start of largest cluster in [art2

and its end.

extract only the elements, i.e., w/o the labels.

First join the two largest clusters \index{Partition@{Partition}!purityScore@{purityScore}}
\index{purityScore@{purityScore}!Partition@{Partition}}
\subsubsection{\setlength{\rightskip}{0pt plus 5cm}vector$<$ double $>$ Partition::purity\-Score ({\bf Partition} $\ast$ {\em part2})}\label{classPartition_a37}


Calculates both purity score values againts the given partition p2. 

Normalize by the reference partition

Normalize by the target partition \index{Partition@{Partition}!vipp@{vipp}}
\index{vipp@{vipp}!Partition@{Partition}}
\subsubsection{\setlength{\rightskip}{0pt plus 5cm}void Partition::vipp ({\bf Partition} $\ast$ {\em part2})}\label{classPartition_a32}


Calculates and prints vi distance againts given partition part2. 

For each cluster cla and

clb , find the 

number of items they have in common.

VI make sense only if number items is equeal.

Using the same normalization for both entropies

Using the same normalization for both entropies

is tantamout to assuming each item from partition1 not present in partition2 is a singleton in partition2 

The documentation for this class was generated from the following files:\begin{CompactItemize}
\item 
Partition.h\item 
Partition.cc\end{CompactItemize}
